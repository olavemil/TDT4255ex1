\chapter{Introduction}
\section{Description of the exercise}

\subsection{Description introduction}

In this second assignment, we have extended the processor from the previous
assignment by changing the datapath into a pipeline. This had the consequence
that pipelines had to be added to the design, as well as write a new control
module to support the pipeline processing.
\paragraph*{}
Additionally, we have
implemented a hazard detection unit and a forwarding unit to guard the
processor from hazards by using correction techniques. The forwarding unit
sends data yet to be written to memory to previous pipeline stages where later
instructions need the current values. The hazard detection unit inserts
``bubbles'' into the pipeline, stalling a selected stage for the duration of a
pipeline clockcycle. It does this to stop a new instruction reading from a
register which a previous instruction is currently writing to.
\paragraph*{}
Both these units have but one intention and consequence, and that is
implementing a more efficient design of the pipelined processor by avoiding
stalls and hazards.

\subsection{Assignment requirements}
\emph{The below text has been copied from the Course Compendium
\cite{compendium}.}\newline
``The major requirement of this assignment is a simple 5-stage pipelined
processor. In general, the processor has the same functional requirements as in
the previous assignment. Additionally, you need to implement different hazard
detection and correction techniques. You will also use the same test set up. To
help you on the way, we have made a suggestion from which you can work on. It is
wise to make a processor which relies on the design from the previous assignment
so that you can reuse the test benches and test programs.''
\paragraph*{}
In consequence, we interpreted this to mean that we needed the following:
\begin{enumerate}
	\item To design a pipelined processor.
	\begin{enumerate}
		\item With five pipeline stages.
	\end{enumerate}
	\item The processors functionality will be based on the functionality of a
``MIPS'' processor.
	\item A Forwarding unit to deal with some of the hazards a pipelined
processor brings.
	\item A Hazard Detection unit to deal with the hazards that occur when a
processor reads and writes in two different pipeline stages.
\end{enumerate}


\section{Approach to the task}\label{intro:approach}
Our approach to the task has been an iterative approach. Slide 58 on the
\emph{Pipelining} \cite{slides-6} course lecture illustrates a pipelined design
whose components resemble the previous assignment in this course very closely.
Due to this, we decided to start by converting the functioning processor design
from the previous assignment into a pipelined processor based on the design on
the abovementioned diagram. The diagram is taken from the book
\cite{patterson12}, on page 362.
\paragraph*{}
Following the approach of the authors through the chapters of the book, we had
few problems implementing the first version of our processor design, and
continued to work towards a more complex design including hazard controls by way
of forwarding and stalling. Slide 57 from lecture \cite{slides-7} illustrates
the functional architecture of our final processor design. This design builds
upon the previous one with several improvements, which include the forwarding
unit and the hazard control unit.
\paragraph*{}
Finally, when the design was completed, we tested it out in the lab to discover
any errors not previously found in our design (and subsequently fixing them).


\section{Challenges with pipelined design}

Working on the design and implementation of a pipelined processor compared to a
Multi-Cycle processor, differs mainly in its level of complexity.
\paragraph*{}
There are many reasons for why a pipelined processor is more complex, but we
struggled mostly with the signals propagations, the pipeline stages
interlocking, and the added hazard control logic.
\paragraph*{}
Since splitting the components into different pipelines (yet having said stages
working in an interlocked fashion), results in having just as many, if not more
signals to keep in mind and under control than a Multi-Cycle processor. And due
to the pipelines working in an interlocked fashion, another level of difficulty
is introduced when making sure that all the signals propagate throughout the
pipeline correctly.
\paragraph*{}
Implementing the added hazard control logic through the \emph{forwarding} and
\emph{hazard detection} components also added more complexity to the processor
design as a whole, instead of simply being new components containing complex
circuitry and logic.
