\chapter{Introduction}
\section{Description of the exercise}

For this assignment, the design from the previous assignment was to be extended
with pipelining, hazard control and related optimizations. These are all steps
that aim to improve the efficiency and throughput of the processor. While much
of the structure remained the same, many of the components had to be modified,
or re-written entirely.

\subsection{Exercise requirements}
\emph{The task as stated in the compendium \cite{compendium}.}\newline
\textbf{\emph{``The major requirement of this assignment is a simple 5-stage
pipelined processor. In general, the processor has the same functional
requirements as in the previous assignment. Additionally, you need to implement
different hazard detection and correction techniques.''}}
\paragraph*{}
The following ordering of goals were:
\begin{enumerate}
	\item Divide the processor into the five pipeline stages.
	\item Re-use code that that is compatible with the new pipeline design.
	\item Implement forwarding.
	\item Implement data hazard detection and correction with stalling.
	\item Implement control hazard avoidance with improved branching and
flushing.
\end{enumerate}


\section{Approach to the task}\label{intro:approach}
Our approach to the task has been an iterative approach. Firstly, we aimed for a
design based on a diagram on slide 58 from a Course lecture \cite{slides-6}.
This slide illustrates a pipelined design which components resemble previous
assignment in this course very closely. Due to this, we decided to start by
converting the functioning processor design from the previous assignment into a
pipelined processor based on the design on the abovementioned diagram.
\paragraph*{}
This diagram is taken from the book \cite{patterson12}, on page 362. Since the
design has been thought through by the books authors, and is well described in
their book, we had few problems implementing this first version of our processor
design.
\paragraph*{}
Following the approach of the authors through the chapters of the book, we
started to work towards a more complex design including hazard controls by way
of forwarding and stalling. Slide 57 from lecture \cite{slides-7} illustrates
the functional architecture of our final processor design. This design builds
upon the previous one with several improvements, which include the forwarding
unit and the hazard control unit.
\paragraph*{}
Finally, when the design was completed, we tested it out in the lab to discover
any errors not previously found in our design (and subsequently fixin them).


\section{Challenges with pipelined design}

Working on the design and implementation of a pipelined processor compared to a
Multi-Cycle processor, differs mainly in its level of complexity.
\paragraph*{}
There are many reasons for why a pipelined processor is more complex, but we
struggled mostly with the signals propagations, the pipeline stages
interlocking, and the added hazard control logic.
\paragraph*{}
Since splitting the components into different pipelines (yet having said stages
working in an interlocked fashion), results in having just as many, if not more
signals to keep in mind and under control than a Multi-Cycle processor. And due
to the pipelines working in an interlocked fashion, another level of difficulty
is introduced when making sure that all the signals propagate throughout the
pipeline correctly.
\paragraph*{}
Implementing the added hazard control logic through the \emph{forwarding} and
\emph{hazard detection} components also added more complexity to the processor
design as a whole, instead of simply being new components containing complex
circuitry and logic.
