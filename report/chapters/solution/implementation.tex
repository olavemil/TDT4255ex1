Similarly to the last exercise, a hierarchical structure was used where the
processor had each stage as a component. This was done to provide abstraction,
as well as simplifying the implementation.
\paragraph*{}
The VHDL component called \emph{pipeline\_stage1} consists of the abstracted
pipeline stages 0 and 1, since each of these are relatively small in their own
right. Stage 2 was mostly implemented how it was designed, but the hazard
detection unit and control unit were placed inside this stage. This was done
since most signals connecting to these units go to stage 2.
\paragraph*{}
Stage 5 did not get its own VHDL component, since the program counter was
written into the stage 1 component. The rest of the stage 5 logic fit better
inside the processor component. The forwarding unit was placed as a component in the
processor file to avoid further complexity in the VHDL components.

\subsection{Processor core}
The processor core itself was implemented as a messenger between the pipeline
stages, similarly to how toplevel works with respect to the core, com and mem
units.
\paragraph*{}
To make it obvious where a signal came from each was prefixed with
``\emph{stage\_\#\_out\_}'', since many of the signals had the same name in
different stages. In addition, all component definitions and port maps were
grouped and documented according to where they were connected, making it easy to
locate a desired signal. These steps helped a great deal during development, but
may have been neglected during testing/bugfixing.
\paragraph*{}
Since it was unclear whether memory access would consume a processing cycle or
 need a pipeline register, IMEM was connected directly to stage one, and
 instructions were then sent from a port in stage 1 onward to stage 2. This made
 it easy to implement or bypass the pipeline register as needed.
 
\subsection{Control unit}
The control unit was changed quite a lot from the previous assignment. Since it was used to control the multi-cycle functionality of the previous exercise, it could now be implemented without any states. Since it no longer needed to account for states, the control unit was reduced down to a compnent that no longer depended on the internal clock. A process inside the control unit responds to a change in the opcode and outputs the correct control signals for the new instruction. These new values are stored in the pipeline registers and propagated through the core. 

\subsection{Hazards}
We made an attempt at implementing hazard detection system such as (and
including) forwarding, stalling and flushing. The forwarding unit is based on
the description given in section 4.7 of \cite{patterson12}. Initially only the
forwarding unit given in the schematic was implemented, but recognizing the
need to forward values to the branch check after the register file, an
additional forwarding unit was instantiated.
\paragraph*{}
Dynamic branch prediction was postponed to give more time for testing, and
branches are in the design assumed to fall through. If this fails, the pipeline
is stalled, but the optimization of moving the branch check to stage 2 has been
implemented. Flushing the current instruction based on this branch value leads
to a combinatorial loop, and although it can probably be easily fixed, this
flushing was commented out during testing.
\paragraph*{}
A hazard detection unit also takes care of data hazards relating to reading
values from registers in use at the ALU stage (stage 3, RAW-hazards). The
hazard detection unit stalls the pipeline when the instruction passing from
stage 2 and into stage 3 involves a register read. If this is the case, and said
instruction reads from one of the two registers in use in stage 2, the unit will
enable the stall signal passing into stage 1. If the stall signal in stage 1 is
enables at clockcycle $n$, this has the consequence of all subsequent $n+1$
stages repeating their action at clockcycle $n+1$, creating an instruction
``bubble'' at stage $n+1$.
