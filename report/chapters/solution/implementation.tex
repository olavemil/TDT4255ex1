Similarly to the last exercise, a hierarchical structure was used where the
processor had each stage as a component. This was done to provide abstraction,
as well as simplifying the implementation. The VHDL component called
\emph{pipeline\_stage1} consists of the abstracted pipeline stages 0 and 1,
since each of these are relatively small in their own right. Stage 2 was
mostly implemented how it was designed, but the hazard detection unit and
control unit were placed inside this stage. This was done since most signals
connecting to these units go to stage 2. Stage 5 did not get its own VHDL
component, since the program counter was written into the stage 1 component. The
rest of the stage5 logic fit better inside the processor component. The
forwarding unit is a component in the processor component to avoid further
complexity in the VHDL components.

\subsection{Processor core}
The processor core itself was implemented as a messenger between the pipeline
stages, similarly to how toplevel works with respect to the core, com and mem
units. To make it obvious where a signal came from each was prefixed with
``stage\_\#\_out\_'', since many of the signals had the same name in different
stages. In addition, all component definitions and port maps were grouped and
documented according to where they were connected, making it easy to \textbf{(Mener du detect eller "find" her? $\Rightarrow$)} detect a desired signal.
\textbf{TODO: decide on whether mem access will belong to components and be registered, or belong to processor and go unregistered, base on testing/timing results.}

\subsection{Control unit}
\textbf{TODO: changes to the control unit from last time}
\subsection{Hazards}
\textbf{TODO: choices regarding implementation of hazard control, including, but not limited to, forwarding, stalling, flushing and branching.}
\emph{Forwarding data from MEM to EX with the forwarding should fix most issues.
\\
We have implemented the simple branch prediction described in the book
(assume it will fall through), and the optimization moving the branch logic to
stage 2 (ID). We do not currently forward values to the branching check!
\\
Hazard detection unit should stall the pipeline necessary. More info on this
when it is confirmed or fixed.}

\subsection{????}