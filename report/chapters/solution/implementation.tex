Similarly to the last exercise, a hierarchical structure was used where the
processor had each stage as a component. This was done to provide abstraction,
as well as simplifying the implementation.
\paragraph*{}
The VHDL component called \emph{pipeline\_stage1} consists of the abstracted
pipeline stages 0 and 1, since each of these are relatively small in their own
right. Stage 2 was mostly implemented how it was designed, but the hazard
detection unit and control unit were placed inside this stage. This was done
since most signals connecting to these units go to stage 2. Stage 5 did not get
its own VHDL component, since the program counter was written into the stage 1
component. The rest of the stage 5 logic fit better inside the processor
component. The forwarding unit is a component in the processor component to
avoid further complexity in the VHDL components.

\subsection{Processor core}
The processor core itself was implemented as a messenger between the pipeline
stages, similarly to how toplevel works with respect to the core, com and mem
units.
\paragraph*{}
To make it obvious where a signal came from each was prefixed with
``\emph{stage\_\#\_out\_}'', since many of the signals had the same name in
different stages. In addition, all component definitions and port maps were
grouped and documented according to where they were connected, making it easy to
locate a desired signal.\newline
\textbf{TODO: decide on whether mem access will belong to components and be registered, or belong to processor and go unregistered, base on testing/timing results.}

\subsection{Control unit}
\textbf{TODO: changes to the control unit from last time}
\subsection{Hazards}
A hazard detection system including forwarding, stalling and flushing was attempted implemented. The forwarding unit is based on the description given in section 4.7 of \ref{book}. A process that forwards values through the register file was also implemented, as is mentionen briefly on page  378.\\
Dynamic branch prediction was postponed to give more time for testing, and branches are assumed to fall through. If this fails the pipeline is stalled, but the optimization moving the branch check to stage 2 has been implemented. Flushing the current instruction base on this branch value leads to a combinatorial loop, and although it can probably be easily fixed, this flushing was commented out during testing.\\
A hazard detection unit also takes care of data hazards relating to loads. \textbf{TODO:CC}
Hazard detection unit should stall the pipeline necessary. More info on this when it is confirmed or fixed.

\subsection{????}