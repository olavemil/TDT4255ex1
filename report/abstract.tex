After the proliferative growth of complexity anticipated by Moore's law, the
processor technology has had to struggle more and more in finding ways to
increase throughput without hitting their many limitations due to their
complexity. Pipelining is a technique to utilize the components in a processor
much more efficiently, by reducing downtime as much as possible. This through
identifiying seperable stages which can work independently of each other, and
segregating these so that (under ideal circumstances) no stage in the pipeline
has any inactive circuits at any given time.
\paragraph*{}
This technique has been implemented successfully throughout the industry, and
there are mainstream processors on the market without pipelining to this day.
\paragraph*{}
In this assignment, the goal has been to produce a five-stage pipelined MIPS
processor. This has been realized through current industry tools such as
Xilinx's Integrated Software Environment Project Navigator and Platform Studio,
which are both part of the Xilinx Design Suite \cite{xilinx-ise}
\paragraph*{}
The goal of this assignment has been for us students to learn how pipelined
processors work, and to gain familiarity with the current industry tools used
for the design of such processors, by way of implementing it on an embedded
device containing an FPGA chip, the Xilinx Spartan-6 \cite{spartan-6}.
\paragraph*{}
For this assignment, the suggestions in the book \cite{patterson12}, the Course
\cite{course} lectures and slides, and the Compendium \cite{compendium}.