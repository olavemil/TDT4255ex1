Our approach to the task has been an iterative approach. Firstly, slide 58 from
lecture \cite{slides-6} illustrates a pipelined design which resembles the
previous assignment (chapter four in the Compendium \cite{compendium}) in this
course very closely. Due to this fact, we decided to convert the functioning
processor designed in the previous assignment into a pipelined processor based
on the design on slide 58 \cite{slides-6}.
\paragraph*{}
This picture is taken from the book \cite{patterson12}, on page 362. Due to this
design having been thought out by the books authors, and being well described in
said book, we had few problems implementing this first version of our processor
design.
\paragraph*{}
Following the chapters of the book, we started to work towards a more complex
design including hazard controls through forwarding and stalling. Slide 57 from
lecture \cite{slides-7} illustrates the basic functionality of our final
processor design. This design builds upon the previous one with several
improvements, which all added to the complexity of the design as a whole.
\paragraph*{}
Finally, when the design was completed, we tested it out in the lab.

\textbf{Legge til noe om at det var vanskelig å jobbe med dette faget og TDT4295
på en gang? Kanskje det burde gå i diskusjon, vet ikke. (Bruker dette istedenfor
for å skrive "TODO's").}
