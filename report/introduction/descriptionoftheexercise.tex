\section{Description of the exercise}

The text in the two following subsections have been copied from the Course
Compendium \cite{compendium}.

\subsection{Description introduction}

In the second assignment, you will extend the processor from previous
assignment by changing the datapath to a pipeline. This means that you will need
to add pipeline registers and make a new control module that supports pipeline
processing. Additionally you have to optimize its performance by implementing
different hazard detection and correction techniques. Some of these techniques
include, but are not limited to, data forwarding and pipeline interlocks that
stall the pipeline when necessary. Furthermore, you can implement different
optimization techniques to improve the performance of your pipelined processor.

\subsection{Assignment requirements}

The major requirement of this assignment is a simple 5-stage pipelined
processor. In general, the processor has the same functional requirements as in
the previous assignment. Additionally, you need to implement different hazard
detection and correction techniques. You will also use the same test set up. To
help you on the way, we have made a suggestion from which you can work on. It is
wise to make a processor which relies on the design from the previous assignment
so that you can reuse the test benches and test programs.
